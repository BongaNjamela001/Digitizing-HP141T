% ----------------------------------------------------
% Introduction
% ----------------------------------------------------
\documentclass[class=report,11pt,crop=false]{standalone}
\input{../Style/ChapterStyle.tex}
\makenoidxglossaries

\newacronym{adc}{ADC}{\textbf{A}nalog-to-\textbf{D}igital \textbf{C}onverter}
\newacronym{radar}{RADAR}{Radio Detection and Ranging}
\newacronym{avm}{AvM}{\textbf{A}verage \textbf{M}ode}  
\newacronym{crt}{CRT}{\textbf{C}athode \textbf{R}ay \textbf{T}ube}
\newacronym{dft}{DFT}{\textbf{D}igital \textbf{F}ourier \textbf{T}ransform}
\newacronym{eda}{EDA}{\textbf{E}lectronic \textbf{D}esign \textbf{A}utomation}
\newacronym{fft}{FFT}{Fast Fourier Transform}
\newacronym{fpga}{FPGA}{Field Programmable Gate Array}
\newacronym{hp}{HP}{\textbf{H}ewlett-\textbf{P}ackard Company}
\newacronym{if}{IF}{\textbf{I}ntermediate \textbf{F}requency}
\newacronym{lcd}{LCD}{\textbf{L}iquid \textbf{D}isplay}
\newacronym{lo}{LO}{\textbf{L}ocal \textbf{O}scillator}
\newacronym{phm}{PHM}{\textbf{P}eak \textbf{H}old \textbf{M}ode}
\newacronym{rbw}{RBW}{\textbf{R}esolution \textbf{B}andwidth}
\newacronym{rf}{RF}{\textbf{R}adio \textbf{F}requency}
\newacronym{rwm}{RwM}{\textbf{R}a\textbf{w} \textbf{M}ode}
\newacronym{sa}{SA}{\textbf{S}ignal/\textbf{S}pectrum \textbf{A}nalyzer}
\newacronym{sdk}{SDK}{\textbf{S}oftware \textbf{D}evelopment \textbf{K}it}
\newacronym{vsa}{VSA}{\textbf{V}ector \textbf{S}pectrum \textbf{A}nalyzer}


\begin{document}
	% ----------------------------------------------------
	\chapter{Appendix}

	\section{Graduate Attributes}
		
	\begin{table}[ht!]
		\centering
		\begin{tabular}{|c|m{15em}|m{20em}|}
			\hline
			\textbf{GA} & \textbf{Requirement} & \textbf{Justification and Section in Report} \\
			\hline
			1 & \textbf{Problem-solving} &
			The project focuses on digitizing and modernizing the HP141T spectrum analyzer by identifying engineering challenges such as analog signal emulation, safe ADC interfacing, and display integration. Problem-solving was guided by engineering fundamentals to replicate the HP141T's behavior while addressing obsolescence. \textit{See Chapters 2 and 3.} \\
			\hline
			4 & \textbf{Investigations, experiments and data analysis} &
			The project involved unit, integration, and acceptance testing of the emulator and digitized system. Measurements from analog signals were sampled, processed, and evaluated using oscilloscope and data analysis tools. Results were used to confirm emulation fidelity. \textit{See Chapter 5.} \\
			\hline
			5 & \textbf{Use of engineering tools} &
			Modern tools such as KiCad (PCB design), STM32CubeIDE (embedded development), Python (data processing and GUI), and the Picoscope software were used throughout. These tools enabled simulation, design validation, and visualization. \textit{See Chapter 3.} \\
			\hline
			6 & \textbf{Professional and technical communication} &
			Weekly Microsoft Teams meetings were held with the supervisor, and updates were communicated clearly. Project progress, testing results, and implementation steps were documented in the report. A full BoM and Gantt chart were created. \textit{See Chapter 1 and Appendices.} \\
			\hline
			8 & \textbf{Individual work} &
			All major system components—hardware design, firmware development, data processing, and GUI rendering—were independently developed. Feedback was incorporated iteratively throughout the project. \textit{See entire report.} \\
			\hline
			9 & \textbf{Independent learning ability} &
			The project required in-depth understanding of legacy analog circuitry, signal conditioning, ADC interfacing, and embedded development. This knowledge was self-acquired through datasheets, application notes, and literature. \textit{See Chapter 2 and 3.} \\
			\hline
		\end{tabular}
		\caption{Summary of Graduate Attributes Achieved in the Digitized HP141T Project}
		\label{tab:graduate-attributes-hp141t}
	\end{table}
	
	
	\ifstandalone
	\bibliography{../Bibliography/References.bib}
	\printnoidxglossary[type=\acronymtype,nonumberlist]
	\fi
\end{document}
% ----------------------------------------------------