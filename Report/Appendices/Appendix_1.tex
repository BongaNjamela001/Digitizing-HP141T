% ----------------------------------------------------
% Introduction
% ----------------------------------------------------
\documentclass[class=report,11pt,crop=false]{standalone}
\input{../Style/ChapterStyle.tex}
\makenoidxglossaries

\newacronym{adc}{ADC}{\textbf{A}nalog-to-\textbf{D}igital \textbf{C}onverter}
\newacronym{radar}{RADAR}{Radio Detection and Ranging}
\newacronym{avm}{AvM}{\textbf{A}verage \textbf{M}ode}  
\newacronym{crt}{CRT}{\textbf{C}athode \textbf{R}ay \textbf{T}ube}
\newacronym{dft}{DFT}{\textbf{D}igital \textbf{F}ourier \textbf{T}ransform}
\newacronym{eda}{EDA}{\textbf{E}lectronic \textbf{D}esign \textbf{A}utomation}
\newacronym{fft}{FFT}{Fast Fourier Transform}
\newacronym{fpga}{FPGA}{Field Programmable Gate Array}
\newacronym{hp}{HP}{\textbf{H}ewlett-\textbf{P}ackard Company}
\newacronym{if}{IF}{\textbf{I}ntermediate \textbf{F}requency}
\newacronym{lcd}{LCD}{\textbf{L}iquid \textbf{D}isplay}
\newacronym{lo}{LO}{\textbf{L}ocal \textbf{O}scillator}
\newacronym{phm}{PHM}{\textbf{P}eak \textbf{H}old \textbf{M}ode}
\newacronym{rbw}{RBW}{\textbf{R}esolution \textbf{B}andwidth}
\newacronym{rf}{RF}{\textbf{R}adio \textbf{F}requency}
\newacronym{rwm}{RwM}{\textbf{R}a\textbf{w} \textbf{M}ode}
\newacronym{sa}{SA}{\textbf{S}ignal/\textbf{S}pectrum \textbf{A}nalyzer}
\newacronym{sdk}{SDK}{\textbf{S}oftware \textbf{D}evelopment \textbf{K}it}
\newacronym{vsa}{VSA}{\textbf{V}ector \textbf{S}pectrum \textbf{A}nalyzer}


\begin{document}
	% ----------------------------------------------------
	\chapter{Appendix}
	
	\section{Data Acquisition Subsystem Code}
	
	\subsection{\acrshort{das} Macros}
	
	The \acrfull{das} is implemented on the STM32H723ZG microcontroller using embedded C and the STM32 HAL library to sample the horizontal, vertical, and pen-lift signals via ADC1 pins PF10, PF9, and PF8, respectively. A 3-channel \acrshort{dma} stream is used to continuously transfer 16-bit ADC samples into a 402-element buffer, with the \texttt{DMA2\_Stream0\_IRQHandler} managing buffer events via HAL \acrshort{dma} interrupt handling.
	
	\begin{lstlisting}[language=C, label={lst:das-adc-macros}, caption={Showing macros for the \acrshort{adc} channels, \acrshort{dma} and the associated interrupts for handling streaming.}]
		/* ADC Channels */
		#define ADCx_CHANNEL                    ADC_CHANNEL_8  // V_x
		#define ADCx_CHANNEL2                   ADC_CHANNEL_7  // V_y
		#define ADCx_CHANNEL3                   ADC_CHANNEL_6  // V_z
		
		/* DMA Configuration */
		#define ADCx_DMA_CHANNEL                DMA_REQUEST_ADC1 // Master ADC1
		#define ADCx_DMA_STREAM                 DMA2_Stream0
		#define ADCx_DMA_IRQn                   DMA2_Stream0_IRQn
		#define ADCx_DMA_IRQHandler             DMA2_Stream0_IRQHandler
		
		ADC_HandleTypeDef    AdcHandleMaster;
		ADC_HandleTypeDef    AdcHandleSlave;
		
		#define ADC_BUF_SIZE 802 // 802 samples (401 samples/channel)
		volatile uint16_t adc_buffer[ADC_BUF_SIZE] = {0}; // 16-bit for interleaved data
		
		volatile int adc_error_flag = 0;
	\end{lstlisting}

	\subsection{Implementing Direct Memory Access for \acrshort{das}}
	
	The \texttt{adc\_init} function configures ADC1 and ADC2 in double-interleaved mode for continuous 16-bit sampling across three channels, with DMA handling enabled for real-time data acquisition. Although the system is configured to output 32-bit words, the use of a \texttt{uint16\_t} buffer relies on the HAL to extract the lower 16 bits of each sample.
		
	\begin{lstlisting}[language=C, label={lst:das-spectrogram-init-code}, caption={Code snippet of the \texttt{adc\_init} function which configures the double-interleaved operation of the \acrshort{adc}s.}]
		int adc_init(void) {
			ADC_ChannelConfTypeDef sConfig;
			ADC_MultiModeTypeDef multiMode;
			//...
			/* Configure ADC2 (Slave) */
			AdcHandleSlave.Instance = ADCx_SLAVE;
			AdcHandleSlave.Init = AdcHandleMaster.Init; // Same config as master
			if (HAL_ADC_Init(&AdcHandleSlave) != HAL_OK) {
				adc_error_handler();
				return -1;
			}
			
			/* Configure double-interleaved mode */
			multiMode.Mode = ADC_DUALMODE_INTERL_FAST; // Fast interleaved mode
			multiMode.DualModeData = ADC_DUALMODEDATAFORMAT_32_10_BITS;
			multiMode.TwoSamplingDelay = ADC_TWOSAMPLINGDELAY_1CYCLE;
			if (HAL_ADCEx_MultiModeConfigChannel(&AdcHandleMaster, &multiMode) != HAL_OK) {
				adc_error_handler();
				return -1;
			}
			//...
		}
	\end{lstlisting}

	\subsection{Reading Interleaved Data}
	
	\acrshort{adc} data stored in the \texttt{adc\_spectrogram} structure is processed in real-time using \acrshort{dma} half-complete callbacks, such as \texttt{HAL\_ADC\_ConvHalfCpltCallback}. This function reads interleaved $\texttt{V}_\texttt{x}$, $\texttt{V}_\texttt{y}$, and $\texttt{V}_\texttt{z}$ values from the buffer, scales them based on 16-bit resolution, and applies logic for plotting and pen-lift detection using a threshold at mid-scale (32768), with adjustments made for signal inversion and normalization as needed.
	
	\begin{lstlisting}[language=C, label={lst:das-callbacks}, caption={Showing conversion callbacks for processing \acrshort{adc} data when the \acrshort{dma} fills half.}]
		void HAL_ADC_ConvHalfCpltCallback(ADC_HandleTypeDef* AdcHandle) {
			for (int j = 0; j < ADC_BUF_SIZE/6; ++j) {
				uint32_t x_val = adc_buffer[j*3];     // V_x
				uint32_t y_val = adc_buffer[j*3+1];   // V_y
				uint32_t z_val = adc_buffer[j*3+2];   // V_z
				x_val = (x_val * adc_spectrogram->npoints) / 65536; // 16-bit scaling
				y_val = adc_spectrogram->size_y - (adc_spectrogram->size_y * y_val) / 65536;
				z_val = z_val > 32768 ? 1 : 0; // Threshold at mid-scale (~1.65 V)
				if (x_val < 5) prev_x = 0;
				if (x_val > prev_x && z_val == 0) {
					if (pending_normalization) {
						adc_spectrogram->data_normal[x_val] = adc_spectrogram->size_y/2 - y_val;
					}
					adc_spectrogram->data[x_val] = y_val + adc_spectrogram->data_normal[x_val];
					prev_x = x_val;
				}
			}
		}
	\end{lstlisting}
	
	\subsection{Enabling \acrshort{adc}'s and Peripherals on STM32H723ZG}
	
	
	The \texttt{HAL\_MspInit} function is implemented to perform essential system-level initializations for the \acrshort{das} using the STM32 \acrshort{hal} framework. It enables clocks for \acrshort{adc}s, GPIOF, and DMA2, and configures the required \acrshort{gpio} pins as analog inputs without pull-up resistors. The function also sets up \acrshort{dma} with 16-bit alignment in circular mode and links it to the \acrshort{adc}, while configuring \acrshort{nvic} interrupts to handle \acrshort{dma} transfer events efficiently.
	
	\begin{lstlisting}[language=C, label={lst:das-callbacks}, caption={Showing conversion callbacks for processing \acrshort{adc} data when the \acrshort{dma} fills half.}]
	void HAL_ADC_MspInit(ADC_HandleTypeDef *hadc) {
		GPIO_InitTypeDef GPIO_InitStruct = {0};
		static DMA_HandleTypeDef hdma_adc;
		
		if (hadc->Instance == ADCx_MASTER) {
			/* Enable clocks */
			ADCx_CLK_ENABLE();0
			ADCx_CHANNEL_GPIO_CLK_ENABLE();
			DMAx_CLK_ENABLE();
			//...
			/* Configure GPIO pins */
			GPIO_InitStruct.Pin = ADCx_CHANNEL_PIN | ADCx_CHANNEL_PIN2 | ADCx_CHANNEL_PIN3;
			GPIO_InitStruct.Mode = GPIO_MODE_ANALOG;
			GPIO_InitStruct.Pull = GPIO_NOPULL;
			//...
			hdma_adc.Init.PeriphDataAlignment = DMA_PDATAALIGN_HALFWORD; // 16-bit
			hdma_adc.Init.MemDataAlignment = DMA_MDATAALIGN_HALFWORD;
			hdma_adc.Init.Mode = DMA_CIRCULAR;
			//...
			__HAL_LINKDMA(hadc, DMA_Handle, hdma_adc);
			
			/* Configure NVIC */
			HAL_NVIC_SetPriority(ADCx_DMA_IRQn, 0, 0);
			HAL_NVIC_EnableIRQ(ADCx_DMA_IRQn);
		}
		//..
	\end{lstlisting}
	
	\section{Digital Processing Subsystem Code}
	
	\subsection{Data Formatting}
	
	To ensure compatibility between different sampling devices (such as the STM32H7 and the Picoscope), the spectrogram data structure—originally developed for an STM32F746G board—was adapted to process and store interleaved data from three ADC channels into a series of \texttt{.csv} files. The \texttt{spectrogram\_to\_csv()} function manages file creation, formats the time-stamped voltage samples, and writes them in blocks of 801 points per file to support downstream processing in the \acrshort{dps} and \acrshort{guis} subsystems.
	
	\begin{lstlisting}[language=C, label={lst:das-csv-file}, caption={Code snippet of the function for storing the interleaved sampling data in a \texttt{.csv}.}]
		int spectrogram_to_csv(spectrogram_t *s, const char *base_filename) {
			if (!s || !base_filename) return -1;
			
			const double sample_period_ms = 0.000138889; // 1 / 7.2 MSPS
			int file_index = 0;
			int point_count = 0;
			char filename[64];
			FILE *fp = NULL;
			
			for (int i = 0; i < s->npoints && !pause_flag; ++i) {
				if (point_count == 0) {
					// Open new file
					snprintf(filename, sizeof(filename), "%s%d.csv", base_filename, file_index);
					fp = fopen(filename, "w");
					if (!fp) return -1;
					fprintf(fp, "Time (ms),V_x (V),V_y (V),V_z (V)\n");
				}
				
				double time_ms = i * sample_period_ms;
				float vx = s->data_x[i] * 3.3f / 65535.0f;
				float vy = s->data_y[i] * 3.3f / 65535.0f;
				float vz = s->data_z[i] * 3.3f;
				
				fprintf(fp, "%.8f,%.8f,%.8f,%.8f\n", time_ms, vx, vy, vz);
				
				point_count++;
				if (point_count >= 801 || i == s->npoints - 1) {
					fclose(fp);
					point_count = 0;
					file_index++;
				}
			}
			
			return 0;
		}
	\end{lstlisting}
	
	\subsection{Using Piecewise Cubic Polynomial Interpolation to Smooth Display}
	
	The proposed design implements the Cubic spline interpolation algorithm which uses piecewise cubic polynomials to define a curve within different segments of the data. The code snippet in listing \ref{lst:dps-spline} illustrates how the frequency domain plot of the signal is smoothed to reduce sharp edges that can occur when using linear interpolation. The program plots the \acrshort{crt}-like background grid on the touchscreen, allowing users to analyse the input signal in the tuning section.
	
	\begin{lstlisting}[language=Python, caption={Code snippet for performing polynomial interpolation in the \acrshort{dps} to the spectrogram a smoother appearance.}, label={lst:dps-spline}]
	from scipy.interpolate import CubicSpline
	#...
	# Perform cubic spline interpolation
	def interpolate_data(time_ms, power_db):
		cs = CubicSpline(time_ms, power_db)
		time_smooth = np.linspace(time_ms[0], time_ms[-1], 1000)  # Increase resolution
		power_smooth = cs(time_smooth)
		 plt.clf()
		ax = plt.gca()
		
		# Plot with CRT styling using spline-interpolated data
		ax.plot(time_smooth, power_smooth, color='limegreen', linewidth=2)
		
		# CRT-like background and grid
		ax.set_facecolor('black')
		fig.patch.set_facecolor('black')
		ax.grid(True, color='gray', linestyle='--', linewidth=0.5)
		
		# Grid divisions: 10 (time), 8 (amplitude)
		time_max = time_ms[-1]
		ax.set_xticks(np.linspace(0, time_max, 11))
		ax.set_yticks(np.linspace(0, 80, 9))
		
		# Labels and title
		ax.set_xlabel('Frequency (GHz)', color='white', fontsize=12)
		ax.set_ylabel('Amplitude (dB)', color='white', fontsize=12)
		ax.set_title('HP141T Spectrogram', color='white', fontsize=14)
		ax.tick_params(axis='both', colors='white')
		
		# Set axis limits
		ax.set_xlim(0, time_max)
		ax.set_ylim(0, 80)
		
		canvas.draw()
		logging.info("Frequency domain plot updated successfully with spline interpolation")
	\end{lstlisting}

	\section{Graphical User Interface Code}

	\subsection{Pause Functionality Implemented on the Single-Board Computer}
	The code snippet below is used to flag the STM32H723ZG connected to the single-board \acrshort{rpi} 4B to pause data sampling. The code also illustrates enabled logging for debugging and getting feedback on the user's actions.
	\begin{lstlisting}[language=Python, caption={Code snippet for function associated with the pause function in latex.}, label={lst:guis-pause-function}]
	# Global pause flag
	pause_flag = False
	
	def pause_recording():
		global pause_flag
		pause_flag = True
		status_label.config(text="Recording Paused")
		logging.info("Recording paused by user")
	\end{lstlisting}

	\subsection{Enabling the \acrfull{phm}, \acrfull{avm} and \acrfull{rwm} Display Modes}
	The \acrshort{phm} tracks and displays the maximum value seen for each data point and is updated per scan. The \texttt{hp141t} Python module includes a \texttt{SpectrumProcessor Class} for updating the \texttt{time\_data} with the maximum value at each index. 
	
	\begin{lstlisting}[language=Python, caption={Code snippet for enabling the \acrshort{phm} when the user presses the \texttt{PHM} button on the screen.}, label={lst:guis-phm}]
		 # Update mode-specific data
		if self.mode == 'PHM':
			np.maximum(self.peak_data, power_db, out=self.peak_data)
			return self.peak_data
		elif self.mode == 'AvM':
			if self.scan_count == 1:
				self.avg_data = power_db
			else:
				self.avg_data = (self.avg_data * (self.scan_count - 1) + power_db) / self.scan_count
			return self.avg_data
		else:  # RwM
			self.raw_data = power_db
			return self.raw_data
	\end{lstlisting} 

	\subsection{System Documentation and Tutorial for Users}
	
	A \texttt{HELP} button is included to enhance the usability of the system, in line with SRO6. The tutorial is displayed in a read-only \texttt{Text} widget and explains the system to users.
	
	\begin{lstlisting}[language=Python, caption={Code snippet for creating a pop-up window with a tutorial when the user interacts with the \texttt{HELP} button.}, label={lst:guis-show-help}]
		def show_help():
			"""Display a tutorial window"""
			help_window = tk.Toplevel(root)
			help_window.title("Help - HP141T Spectrogram Tutorial")
			help_window.geometry("400x300")
			help_window.configure(bg='black')
			
			# Tutorial text
			tutorial_text = """
			=== HP141T Spectrogram Tutorial ===
			
			Welcome to the HP141T Spectrogram Display!
			This system visualizes data from the HP141T spectrum analyzer on a touchscreen LCD.
			
			**Purpose:**
			- Displays time-domain spectrogram data (0 to 80 dB amplitude, time sweep).
			- Replicates the CRT display with a green trace and grid.
			
			**Modes:**
			- **Peak Hold Mode (PHM):** Shows the largest value seen at each time point, updated per scan.
			- **Average Mode (AvM):** Displays the running average of values at each time point, updated per scan.
			- **Raw Mode (RwM):** Shows the latest value, overwriting previous data during each scan.
			
			**Usage:**
			- **Mode Selection:** Tap 'Peak Hold Mode', 'Average Mode', or 'Raw Mode' to switch display modes.
			- **Pause:** Tap 'Pause Recording' to stop updates.
			- **Help:** Tap 'Help' to view this tutorial.
			- Navigate using touchscreen gestures if supported by future updates.
			
			**Notes:**
			- Data is loaded from 'spectrogram_data_X.csv' files in /home/pi.
			- Contact support for issues (log at /home/pi/spectrogram_display.log).
			
			Close this window to return to the display.
			"""
			
			# Create and configure text widget
			text_widget = tk.Text(help_window, bg='black', fg='white', font=('Courier', 10), wrap='word')
			text_widget.insert(tk.END, tutorial_text)
			text_widget.config(state='disabled')  # Make read-only
			text_widget.pack(expand=True, fill='both', padx=10, pady=10)
	\end{lstlisting} 
	\ifstandalone
	\bibliography{../Bibliography/References.bib}
	\printnoidxglossary[type=\acronymtype,nonumberlist]
	\fi
\end{document}
% ----------------------------------------------------