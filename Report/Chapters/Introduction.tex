% ----------------------------------------------------
% Introduction
% ----------------------------------------------------
\documentclass[class=report,11pt,crop=false]{standalone}
% Page geometry
\usepackage[a4paper,margin=20mm,top=25mm,bottom=25mm]{geometry}

% Font choice
\usepackage{lmodern}

% Wrap text around image
\usepackage{wrapfig}

% Checkmarks
\usepackage{tikz}

% For algorithms
\usepackage[]{algorithm}
% Pseudocode packages
\usepackage{algpseudocode}

% Table color
% \usepackage{colortbl}

% Multiple rows
\usepackage{multirow}

% Lorem ipsum
\usepackage{lipsum}

% Use IEEE bibliography style
\bibliographystyle{IEEEtran}

% Line spacing
\usepackage{setspace}
\setstretch{1.20}

% Ensure UTF8 encoding
\usepackage[utf8]{inputenc}

% Language standard (not too important)
\usepackage[english]{babel}

% Skip a line in between paragraphs
\usepackage{parskip}

% For the creation of dummy text
\usepackage{blindtext}

% Math
\usepackage{amsmath}

% Lists
\usepackage{enumitem}

% Header & Footer stuff
\usepackage{fancyhdr}
\pagestyle{fancy}
\fancyhead{}
\fancyhead[R]{\nouppercase{\rightmark}}
\fancyfoot{}
\fancyfoot[C]{\thepage}
\renewcommand{\headrulewidth}{0.0pt}
\renewcommand{\footrulewidth}{0.0pt}
\setlength{\headheight}{13.6pt}

% Epigraphs
\usepackage{epigraph}
\setlength\epigraphrule{0pt}
\setlength{\epigraphwidth}{0.65\textwidth}

% Colour
\usepackage{color}
%\usepackage[usenames,dvipsnames]{xcolor}

% Hyperlinks & References
\usepackage{hyperref}
\definecolor{linkColour}{RGB}{77,71,179}
\hypersetup{
    colorlinks=true,
    linkcolor=linkColour,
    filecolor=linkColour,
    urlcolor=linkColour,
    citecolor=linkColour,
}
\urlstyle{same}

% Automatically correct front-side quotes
\usepackage[autostyle=false, style=ukenglish]{csquotes}
\MakeOuterQuote{"}

% Graphics
\usepackage{graphicx}
\graphicspath{{Images/}{../Images/}}
\usepackage{makecell}
\usepackage{transparent}

% SI units
\usepackage{siunitx}

% Microtype goodness
\usepackage{microtype}

% Listings
\usepackage[T1]{fontenc}
\usepackage{listings}
\usepackage[scaled=0.8]{DejaVuSansMono}

% Custom colours for listings
\definecolor{backgroundColour}{RGB}{250,250,250}
\definecolor{commentColour}{RGB}{73, 175, 102}
\definecolor{identifierColour}{RGB}{196, 19, 66}
\definecolor{stringColour}{RGB}{252, 156, 30}
\definecolor{keywordColour}{RGB}{50, 38, 224}
\definecolor{lineNumbersColour}{RGB}{127,127,127}
\lstset{
  language=Matlab,
  captionpos=b,
  aboveskip=15pt,belowskip=10pt,
  backgroundcolor=\color{backgroundColour},
  basicstyle=\ttfamily,%\footnotesize,        % the size of the fonts that are used for the code
  breakatwhitespace=false,         % sets if automatic breaks should only happen at whitespace
  breaklines=true,                 % sets automatic line breaking
  postbreak=\mbox{\textcolor{red}{$\hookrightarrow$}\space},
  commentstyle=\color{commentColour},    % comment style
  identifierstyle=\color{identifierColour},
  stringstyle=\color{stringColour},
   keywordstyle=\color{keywordColour},       % keyword style
  %escapeinside={\%*}{*)},          % if you want to add LaTeX within your code
  extendedchars=true,              % lets you use non-ASCII characters; for 8-bits encodings only, does not work with UTF-8
  frame=single,	                   % adds a frame around the code
  keepspaces=true,                 % keeps spaces in text, useful for keeping indentation of code (possibly needs columns=flexible)
  morekeywords={*,...},            % if you want to add more keywords to the set
  numbers=left,                    % where to put the line-numbers; possible values are (none, left, right)
  numbersep=5pt,                   % how far the line-numbers are from the code
  numberstyle=\tiny\color{lineNumbersColour}, % the style that is used for the line-numbers
  rulecolor=\color{black},         % if not set, the frame-color may be changed on line-breaks within not-black text (e.g. comments (green here))
  showspaces=false,                % show spaces everywhere adding particular underscores; it overrides 'showstringspaces'
  showstringspaces=false,          % underline spaces within strings only
  showtabs=false,                  % show tabs within strings adding particular underscores
  stepnumber=1,                    % the step between two line-numbers. If it's 1, each line will be numbered
  tabsize=2,	                   % sets default tabsize to 2 spaces
  %title=\lstname                   % show the filename of files included with \lstinputlisting; also try caption instead of title
}

% Caption stuff
\usepackage[hypcap=true, justification=centering]{caption}
\usepackage{subcaption}

% Glossary package
% \usepackage[acronym]{glossaries}
\usepackage{glossaries-extra}
\setabbreviationstyle[acronym]{long-short}

% For Proofs & Theorems
\usepackage{amsthm}

% Maths symbols
\usepackage{amssymb}
\usepackage{mathrsfs}
\usepackage{mathtools}

% For algorithms
%\usepackage[]{algorithm2e}

% Spacing stuff
\setlength{\abovecaptionskip}{5pt plus 3pt minus 2pt}
\setlength{\belowcaptionskip}{5pt plus 3pt minus 2pt}
\setlength{\textfloatsep}{10pt plus 3pt minus 2pt}
\setlength{\intextsep}{15pt plus 3pt minus 2pt}

% For aligning footnotes at bottom of page, instead of hugging text
\usepackage[bottom]{footmisc}

% Add LoF, Bib, etc. to ToC
\usepackage[nottoc]{tocbibind}

% SI
\usepackage{siunitx}

% For removing some whitespace in Chapter headings etc
\usepackage{etoolbox}
\makeatletter
\patchcmd{\@makechapterhead}{\vspace*{50\p@}}{\vspace*{-10pt}}{}{}%
\patchcmd{\@makeschapterhead}{\vspace*{50\p@}}{\vspace*{-10pt}}{}{}%
\makeatother
\makenoidxglossaries

\newacronym{radar}{RADAR}{Radio Detection and Ranging}
\begin{document}
% ----------------------------------------------------
\chapter{Introduction}

\section{Background}
Designed and patented in the 1970s, Hewlett-Packard's (HP) high performance plug-in model 8552B and 8555A spectrum analyzers (SA), equipped with the 141T display, remain powerful tools for characterising signals in the frequency domain. The 8552B is particularly convenient for measuring spectra in a wide frequency range between $\SI{20}{\hertz}$ to $\SI{40}{\giga\hertz}$. Another advantage of these spectrum analysers is that a user can broaden frequency requirements by increasing the number of tuning sections. Additionally, the 141T features absolute calibration of amplitude as well as high resolution, sensitivity and a simple display output.  

The shortcoming of the spectrum analyzer, however, is that it uses a cathode ray tube (CRT) display which is prone to degradation after extended periods of usage and is outdated compared to the display on most modern devices. In this project, a single board computer and a liquid crystal display (LCD) touch screen is interfaced with the 141T display unit, thereby, replacing the outdated CRT technology. This allows users to continue to exploit the advantageous capabilities of the spectrum analyzer, such as the wide frequency bandwidth, despite damage to the CRT display. In addition, interfacing a single board computer with the 141T offers improved software-based features for performing frequency analysis.

Specifically, this project aims to develop a new display for the high resolution 8552B intermediate frequency (IF) section equipped with the 8555A spectrum analyzer RF section which can make frequency domain measurements from $\SI{10}{\mega\hertz}$ to $\SI{18}{\giga\hertz}$. The broad scanning frequency bandwidth of this model makes it suitable for spectrum surveillance, RF and EMC field strength analysis with a calibrated antenna \cite{HP}. 

The CRT display subsystem consists of a post-accelerator storage tube with a $\SI{9}{\kilo\volt}$ accelerating potential and aluminized P31 phosphor for producing high trace brightness. When calibrated, the CRT screen can display frequency bandwidths of up to $\SI{2}{\giga\hertz}$ wide. To display the full frequency range with a maximum of $\SI{18}{\giga\hertz}$, the CRT can be calibrated in 10 frequency bands using internal mixing. One of advantage of the 141T over other displays that were manufactured during that time is that more detail can be observed in the spectrum by progressively narrowing frequency width from $\SI{100}{\hertz}/\text{division}$ to $\SI{2}{\kilo\hertz}/\text{division}$. Overall, the 141T consists of a CRT graticule which can plot the frequency domain representation of a signal on a 2D plane with 8 x 10 divisions. 

For this project, the 141T is powered by a $\SI{220}{\volt}$ single-phase source at $\SI{60}{\hertz}$, requiring less than $\SI{225}{W}$ even when plug-ins are connected. To achieve the $\SI{9}{\kilo\volt}$ accelerating potential for deflecting electron beams in producing the CRT display, the device uses a step-up transformer and transistorized oscillator. The main disadvantage of having to increase the accelerating potential in a CRT display system is that the performance of electronic voltage regulation components such as capacitors, diodes and resistors can degrade over time. 

Another challenge of using a phosphor CRT display is the effect of persistence on the saccadic information transfer which can lead to bias in experimental results \cite{wolf1997}. This effect of persistence on experimental results is of particular interest to frequency domain analysis since displayed signals include noise from the environment which can make it difficult to extract accurate frequency information from plots. For the Model 141T, the persistence varies from the natural persistence of P31 phosphor ($\SI{0.1}{\second}$) to a maximum of $\SI{15}{\second}$ when the device is operating in the maximum writing rate mode. Therefore, phosphor persistence in the CRT display can significantly affect the amount of time to acquire measurements as well as the precision of the data extracted from the display.

\section{Objectives}
%\lipsum[1]
The aim of this project is to design a new display with full functionality and computer-aided signal processing features such as signal normalization. The digital display has to be compatible with the voltage outputs that enable analog signals to be plotted by the 141T. The aim of digitizing the signals is to interface measures from the spectrum analyzer with a computer that can perform tasks and store data accordingly. Therefore, a survey of the 8555A RF section and 8552B IF section outputs and available single board computer and touch screen options must be conducted. Furthermore, the project aims to investigate basic XYZ replicas, performing digital signal processing algorithms, how to correctly display signal data on annotated axes depending on available instrument settings.

This report aims to provide:
\begin{itemize}
	\item 
	Characterization of the HP141 display inputs from the 8555A RF section and 8552B IF section
	\item 
	Available options for single-board computer and touchscreens options and the most suitable selection for interfacing with the two spectrum analyzer sections
	\item 
	A design and simulation of interface between the single-board computer which includes analog converter for digitizing outputs from the RF and IF sections
	\item 
	Algorithms for processing the digital signals and performing operations on displayed spectra
	\item 
	Results on the construction, unit tests and integration tests of the improved system for spectral analysis	
\end{itemize}

\section{Project Requirements}

Before detailing system requirements, user requirements were used to scope the project in terms of objectives that are not related to functions and performance. In designing the upgraded or "new" SA, selection of hardware components was conducted with the aim of formulating specifications that successfully fulfill user requirements. Table \ref{tab:user-requirements} summarises the user requirements and gives a short description of the objective.

\begin{table}[!ht]
	\centering
	\caption{User requirements.}
	\label{tab:user-requirements}
	\begin{tabular}{|m{5em}|m{40em}|}
		\hline
		\textbf{ID}	& \textbf{Requirement Description}\\
		\hline
		UR-01	& Display of the new SA must behave like the display of newer generations of signal analyzers, such as the Field Fox. That is, the new SA must achieve more or less the same number data points as the Field Fox (801 points). \\
		\hline
		UR-02	& SA must have the following display modes:
		\begin{enumerate}[label=(\alph*)]
			\item 
			Peak hold mode (PHM) which displays the largest value seen and updated at each scan.
			\item 
			Average mode (AvM) in which each frequency bin's average is updated at each scan. 
			\item 
			Raw mode (RwM) where the latest value is displayed until the next scan and overwriting each value during a scan event.
		\end{enumerate}\\
		\hline
		UR-03	& SA unit must have a suitable vertical resolution based on a $\SI{10}{\decibel}\text{/division}$ in the logarithmic scale.\\
		\hline
		UR-04	& Linear display mode must have low priority.\\
		\hline
		UR-05	& SA display subsystem must have setting markers, similar to the Field Fox analyzer.\\
		\hline
		UR-06	& Design must be capable of storing and recalling traces.\\
		\hline
		UR-07	& Users must be able to change modes by touching the screen and must be able to enter data using a keyboard.\\
		\hline
		UR-08	& All software must load on power up.\\
		\hline
		UR-09	& SA unit must use a single wall wart power source for the display subsystem.\\
		\hline
		UR-10	& Project must develop an HP141T display emulator of horizontal sweep, vertical sawtooth and pen lift state.\\
		\hline
	\end{tabular}
\end{table}

Table \ref{tab:sys-requirements} details system requirements in terms of functions and performance. These requirements were developed after a review of the scope through the formalization of the user requirements in table \ref{tab:user-requirements} above.

\begin{table}[!ht]
	\centering
	\caption{System requirements.}
	\begin{tabular}{|m{5em}|m{40em}|}
		\hline
		\textbf{ID} & \textbf{Requirement Description}\\
		\hline
		SR-01 	& The system must digitize analog outputs from an HP141T 8555A model which performs frequency domain measurements between $\SI{20}{\hertz}$ and $\SI{18}{\giga\hertz}$.\\
		\hline
		SR-02	& Digitized outputs must be interfaced with a single board computer for performing signal processing tasks.\\
		\hline
		SR-03	& The HP141T must interface with a LCD touchscreen display to produce a valid display.\\
		\hline
		SR-04	& The system must include a signal conditioning box for debugging purposes and replicating the outputs of the spectrum analyzer during testing.\\
		\hline
		SR-05   & The system must be simulated using software.\\
		\hline
		SR-06	& The display must include new annotations that take instrument and operator manual inputs into account. \\
		\hline
		SR-07	& The system must include appropriate documentation such as tutorials and operational instructions when using the signal processor and screen.\\
		\hline
	\end{tabular}
\label{tab:sys-requirements}
\end{table}

In addition to the above mentioned system requirements, the project considers the basic configuration parameters that signal analyzers typically provide such as:
\begin{itemize}
	\item 
	Setting the minimum and maximum frequencies to be displayed based on a given center frequency 
	\item 
	Setting the reference amplitude for frequency responses and a span that is suitable for the spectrum analyzer
	\item 
	Setting the frequency resolution according to the passband of the IF filter
	\item 
	Setting the sweep time required to record the full frequency spectrum that is of interest
\end{itemize}

\section{Scope \& Limitations}
%\lipsum[1]
The focus of this report is in the design and implementation of a digitized display for the HP141T that interfaces with a single board computer for storing and manipulating signal data from the oscilloscope. The scope is limited to selection of electronics that are suitable for converting the analog signals from the HP141T that are responsible for displaying signals. The scope only includes a survey of the HP141T circuits and outputs that directly affect how a spectrum is generated with respect to time domain and frequency domain analyses. The paper is not concerned with changing or improving the operational design of the device with respect to its power, amplification and filtering circuitry. 

\section{Report Outline}
%\lipsum[1]
Chapter 2 initiates the report by establishing the general history and theoretical framework for the design and applications of spectrum analyzers in engineering. The same chapter details the previous techniques for converting the output of a SA to digital values that can be manipulated by a processor. Finally, designs of displays are explored in literature to establish an approach to representing the processor output on a LCD screen. 
% ----------------------------------------------------
\ifstandalone
\bibliography{../Bibliography/References.bib}
\printnoidxglossary[type=\acronymtype,nonumberlist]
\fi
\end{document}
% ----------------------------------------------------